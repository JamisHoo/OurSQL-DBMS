\section{功能特点}
    \subsection{底层存储系统}
        独立设计并开发了页式文件系统,支持创建、删除文件,%
        读取、写入指定页,支持动态设置页大小(创建文件时设置)。%
        并配有使用LRU页置换算法的缓冲区管理系统,%
        其接口与页式文件系统完全相同,支持动态设置缓冲区大小。
    \subsection{B+树索引系统}
        独立设计并开发了B+树索引系统。
    \subsection{数据类型}
        支持8位整型(tinyint)、16位整型(smallint)、%
        32位整型(int)、64位整型(bigint)、
        布尔型(bool)、字符串(char, varchar)、%
        \href{http://www.eecs.berkeley.edu/~wkahan/ieee754status/IEEE754.PDF}{IEEE 754}
        32位浮点型(float)、%
        \href{http://www.eecs.berkeley.edu/~wkahan/ieee754status/IEEE754.PDF}{IEEE 754}
        64位浮点型(double),
        其中整形支持有符号(signed)和无符号(unsigned),%
        共计13种数据类型。
    \subsection{聚集函数}
        支持AVG、COUNT、MAX、MIN、SUM五种聚集函数。

        SUM函数对整数类型操作结果自动转换为64位有符号整形,%
        对浮点数操作结果自动转换为64位浮点数。

        COUNT函数返回结果为64位无符号整数。

        AVG函数返回结果自动转化为64位浮点数。

        MAX、MIN函数返回结果为原类型。
    \subsection{多表连接}
        支持多表内连接,与聚集函数、分组查询、结果排序功能兼容。
    \subsection{分组查询}
        支持分组查询,与聚集函数、多表连接、查询结果排序功能兼容。
    \subsection{查询排序}
        支持查询结果排序,与聚集查询、多表连接、分组查询功能兼容。
    \subsection{模糊匹配}
        对所有支持的类型进行
        \href{http://ecma-international.org/ecma-262/5.1/#sec-15.10}{EMACScript}
        语法正则表达式匹配的模糊查询。%
        注意此功能使用匹配模式而非搜索模式。
    \subsection{关系完整性约束}
        对于违反完整性约束的操作,数据库管理系统将拒绝此次操作,%
        必要时进行回滚操作以保证操作的原子性。
        \subsubsection{实体完整性约束}
            每个表最多可创建一个主键,主键取值既不能为空也不能重复。
        \subsubsection{域完整性约束}
            建表时可约束每个属性的取值,%
            约束条件是使用AND连接的条件语句。%
        \subsubsection{参照完整性约束}
            可设置某个属性引用其他表的主键。%
            外键只能取外表主键已有的值,或取空值。
    \subsection{命令解析系统}
        本项目中使用
        \href{http://www.boost.org/doc/libs/1_57_0/libs/spirit/doc/html/index.html}{Boost Spirit}
        进行词法分析和语法分析。%
        Spirit用法灵活、极易扩展。
    \subsection{错误处理系统}
        支持数据库不存在、属性名不存在、数据类型不匹配、字面量溢出、%
        违反域完整性约束、违反参照完整性约束、聚集函数作用于不支持的类型等%
        多达45种语法错误、运行时错误的检测和处理。
    \subsection{交互界面}
        支持从文件读取指令,支持指令断行、支持\#风格的注释。
    \subsection{跨平台}
        项目源代码完全符合C++11标准(ISO/ISC 14882:2011),%
        不使用任何平台相关库。%
        源代码可不经任何修改在Windows、Mac OSX、Linux平台上编译通过且正确运行。%
        运行过程中生成的数据、索引等文件也可以在任何字节序(大端模式或小端模式)%
        相同的平台下交叉使用。


