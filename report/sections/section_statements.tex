\section{语法规则}
    \begin{enumerate}
    \item
    关键字

    不区分大小写。AND, ASC, AVG, BY, CHECK, COUNT, CREATE, DATABASE, DATABASES,
    DELETE, DESC, DESCRIBE, DROP, FALSE, FROM, GROUP, INDEX, INSERT, INTO,
    IS, KEY, LIKE, MIN, MAX, NOT, NULL, ON, ORDER, PRIMARY, SELECT, SET, SHOW,
    SIGNED, SUM, TABLE, TABLES, TRUE, UNSIGNED, UPDATE, USE, VALUES, WHERE.
    \item
    数据类型
    \begin{table}[!hbp]
        \centering
        \caption{数据类型}
        \begin{tabular}{|c|c|}
        \hline
        TINYINT [SIGNED] & 有符号8位整型 \\
        \hline
        TINYINT UNSIGNED & 无符号8位整型 \\
        \hline
        SMALLINT [SIGNED] & 有符号16位整型 \\
        \hline
        SMALLINT UNSIGNED & 无符号16位整型 \\
        \hline
        INT [SIGNED] & 有符号32位整型 \\
        \hline
        INT UNSIGNED & 无符号32位整型 \\
        \hline
        BIGINT [SIGNED] & 有符号64位整形 \\
        \hline
        BIGINT UNSIGNED & 无符号64位整形 \\
        \hline
        BOOL & 布尔型 \\
        \hline
        CHAR, VARCHAR & 字符串,必须显式指定长度 \\
        \hline
        FLOAT & 32位浮点型 \\
        \hline
        DOUBLE & 64位浮点型 \\
        \hline
        \end{tabular}
    \end{table}

    \item
    SQL标识符

    字母或数字或下划线,以字母或下划线开头,且不能为关键字和数据类型。
    \item
    字面量

    整形字面量为十进制整数,浮点型字面量为十进制小数,%
    布尔型字面量取值为TRUE(不区分大小写,下同)或FALSE,空值为NULL,%
    字符串必须使用单引号引用,支持反斜线转义,%
    正则表达式也使用单引号引用。
    \item
    运算符

    运算符左值必须是属性名,右值可以是属性名或字面量。%
    当用在表连接的SELECT语句中,属性名必须显示指定表名。
    
    \begin{table}[!hbp]
        \centering
        \caption{运算符}
        \begin{tabular}{|c|c|}
        \hline
        = & 相等 \\
        \hline
        <>, != & 不相等 \\
        \hline
        > & 大于 \\
        \hline
        < & 小于 \\
        \hline
        >= & 大于等于 \\
        \hline
        <= & 小于等于 \\
        \hline
        IS & 值为空,右值只能是NULL \\
        \hline
        IS NOT & 值不为空,右值只能是NULL \\
        \hline
        LIKE & 正则表达式匹配 \\
        \hline
        NOT LIKE &  正则表达式不匹配 \\
        \hline
        \end{tabular}
    \end{table}
    \item
    CREAETE DATABASE <name>;

    创建名为name的数据库。
    \item
    SHOW DATABASES;

    列出所有数据库。
    \item
    USE <name>;

    打开名为name的数据库。
    \item
    DROP DATABASE <name>;

    删除名为name的数据库。
    \item
    CREATE TABLE <table\_name> (<field\_name> <type>[(<size>)] [NOT NULL] 
                         [,<filed\_name> <type>[(<size>)] [NOT NULL]]* 
                         [, PRIMARY KEY (<field\_name>)]
                         [, CHECK (<conditions>)]
                         [, FOREIGN KEY (<field name>) REFERENCES <table\_name>(<field\_name>)]*
                         );

    创建名为table\_name的表,包含若干个feild,并设置关系完整性约束。
    \item
    SHOW TABLES;

    列出当前数据库的所有表。
    \item
    DESC[RIBE] <name>;
    
    显示名为name的表信息。
    \item
    DROP TABLE <name>;

    删除名为name的表。
    \item
    CREATE INDEX ON <table\_name> (<field\_name>);
    
    在表table\_name的属性field\_name上创建索引。
    \item
    DROP INDEX ON <table\_name> (<field\_name>);

    删除表table\_name的属性field\_name上的索引。
    \item
    INSERT INTO <table\_name> VALUES (<value> [, <value>]*)(, <value> [, <value>]*)*;

    插入一系列记录到表table\_name中。
    \item
    DELETE FROM <table\_name> [WHERE <conditions>];

    从表table\_name中删除所有满足条件的记录。
    \item
    UPDATE <table\_name> SET <field\_name> = <new\_value> 
                     [, <field\_name> = <new\_value>]* 
        WHERE <conditions>;

    更新表table\_name中满足条件的记录的部分字段。
    \item
    SELECT [AVG|SUM|MAX|MIN|COUNT(]<field\_name | *>[)] 
           [, [AVG|SUM|MAX|MIN|COUNT(]<field\_name | *>[)]]* FROM <table\_name> 
                    [WHERE <conditions>]
                    [, GROUP BY <field\_name>]
                    [, ORDER BY <field\_name> [ASC | DESC]];

    从表table\_name中选择满足条件的记录,按照某个属性分组,使用聚集函数,%
    按照某个属性排序, 显示部分属性。
    \item
    SELECT [AVG|SUM|MAX|MIN|COUNT(]<table\_name>.<field\_name>[)] 
           [, [AVG|SUM|MAX|MIN|COUNT(]<table\_name>.<field\_name>[)]]*
        FROM <table\_name> [, <table\_name>]* [WHERE <conditions>]
                                            [, GROUP BY <table\_name>.<field\_name>]
                                            [, ORDER BY <table\_name>.<field\_name> [ASC | DESC]];
    
    多表的内连接。



    \end{enumerate}


